\documentclass[english]{jamk-report}

% -------------------- Start preamble -------------------- %

%%% Document information

% You may ignore or delete the commands ending in "fi" if you don't need a Finnish abstract page

% Title
\title{AAAAAAAAAAA \LaTeX\ TÄTÄ ON MUOKATTU}

% Authors: separate authors with a comma (,)
\author{Tuomo Sipola, Niko Ålander, Karo Saharinen, Samir Puuska, Marko Silokunnas}

% Level: either "Bachelor's thesis" or "Master's thesis"
% In Finnish, either "Opinnäytetyö AMK" or "Opinnäytetyö YAMK"
\level{Bachelor's thesis}
\levelfi{Opinnäytetyö AMK}

% Let this be here, if you have authorized web publication
% Delete or commment out the line below, if you don't want to give web publication permission 
\webpublicationpermission

% Field of studies, most likely "Information and Communication Technology" 
\field{Information and Communication Technology}
\fieldfi{Tietojenkäsittely ja tietoliikenne}

% Degree programme
\programme{Degree Programme in Information and Communication Technology}
\programmefi{Tieto- ja viestintätekniikan tutkinto-ohjelma}

% Document language
\documentlanguage{English}
\documentlanguagefi{englanti}

% Set the date in format YYYY-MM-DD
% If not set, it uses the current date
%\date{1643-01-04}

% Abstract
\abstract{
When completing this form, start from this field, on the row under the instructions, so that the font size remains 11.

The basic structure of the abstract is as follows:

\begin{itemize}
    \item background
    \item task and objectives
    \item implementation method
    \item results
    \item conclusions.
\end{itemize}

In other words, the abstract summarises the work that has been done – not the content of the report. If there is room, the content of the report may be briefly mentioned.
The entire space reserved for the abstract must be used.
The abstract should be written in the past tense and with a passive voice. The text must not refer to the thesis, i.e., the words ‘this thesis’ must not be used.
} % end abstract

% Finnish abstract: Feel free to delete this if you don't need it
\abstractfi{
Aloita tiivistelmän kirjoittaminen tästä. Tiivistelmän perusrakenne on seuraava:

\begin{itemize}
    \item tausta, tehtävä ja tavoitteet
    \item toteutustapa
    \item tulokset
    \item johtopäätökset
\end{itemize}
        
Tiivistelmä tiivistää siis tehdyn työn - ei raportin sisältöä. Jos tilaa jää, voi lyhyesti mainita raportin sisällöstä.
Koko tiivistelmälle varattu tila tulee käyttää.
Tiivistelmä kirjoitetaan imperfektissä ja passiivissa. Tekstissä ei saa viitata opinnäytetyöhön eli ei saa käyttää sanoja "tämä työ".

} % end abstractfi

% Keywords
\keywords{See Project Reporting Instructions, section 4.1.2}
\keywordsfi{Ks. raportointiohje luku 4.1.2}

% Miscellaneous
\miscinfo{For example, the confidentiality marking of the thesis appendix, see Project Reporting Instructions, section 4.1.2}
\miscinfofi{Esim. opinnäytetyön liitteen salassapitoperuste, ks. raportointiohjeen luku 4.1.2}

% Bibliography file
\addbibresource{refs.bib}

% List of acronyms:
\DeclareAcronym{tcp}{
  short = TCP ,
  long  = Transmission Control Protocol ,
  tag = acro
}

% -------------------- Preamble ends -------------------- %

% ------------------- Document starts ------------------- %

\begin{document}



%%% FRONTMATTER

% These elements are mandated by the template
% The only thing you might need to delete is the Finnish abstract page

% Start Roman page numbering
\frontmatter

% Title page
\maketitle

% Abstract pages
% Delete or comment out if you don't need the Finnish one
% Change their order if necessary, or delete both if not needed
\abstractpage
\abstractpagefi

%%% MAINMATTER

% Clear page, start Arabic page numbering
\mainmatter

% Deleting the lines below will remove parts of the document if they are not needed
\tableofcontents  % Include table of contents

% Deleting the lines below will remove parts of the document if they are not needed
\listoffigures    % Include list of figures
\listoftables     % Include list of tables

\clearpage

% Acronym page - remove lines below if not needed
\addcontentsline{toc}{section}{Acronyms}
\printacronyms[include=acro,name=Acronyms]

\clearpage



%%% CONTENTS OF THE THESIS

\section{Introduction}

\textbf{NOTICE! THIS DOCUMENTATION IS OUTDATED.} You can still use the thesis class, as you can just fill in the details in the preamble and then write your thesis below.

Latex is a powerful system for creating professional looking documents, such as reports, thesis's, and
scientific papers.

This document strives to both help you in creating documents with \LaTeX, %https://oppimateriaalit.jamk.fi/raportointiohje/
as well as serve as a template that you can easily adjust to your needs.
This document follows the official JAMK report instructions as closely
as it is practical. Nevertheless, if your class or instructor requires
specific style, you may have to adjust this template. The font selection
is motivated by the on-screen readability factors; text hinting has
been confirmed to work in Adobe Acrobat and several Linux PDF readers.

The full template consists of three separate files: \texttt{template.tex}
is the main TeX file, where you produce your text. All document specific
options are adjustable in this file. \texttt{jamk-report.cls}
is the class file, that provides the basic report style. In most use cases there should be
no reason to adjust the template directly. Finally, \texttt{refs.bib} contains
an example bibliography in bibtex format. You should add your bibtex entries to this file.

While much work has been done to keep necessary adjustments minimal for
standard usage, there are still minor manual tweaks required if the user
wishes to change language. Some packages, such as the \texttt{acro} package,
do not support automatic localization. These exceptions are documented in the
\texttt{template.tex} file. This template currently supports Finnish and English.


\section{General Document Structure}

The first page of the JAMK report format is the title page. It has all the
elements one might expect to find on a title page, i.e. the title, author,
document type, and other descriptors. You should change these variables
in \texttt{template.tex} file.

\subsection{Class Options}

This template is based on the standard \texttt{article} class. The only option
that users should adjust is the language, by changing the documentclass entry
to desired option.

\begin{verbatim}
\documentclass[english]{jamk-report}
\end{verbatim}

\noindent
Paper sizes other than A4 are not supported. It is not advisable to adjust text
size from the default 11p, although the package supports also 10p.

\subsection{Document Variables}

The template requires that the user sets certain variables, such as the title,
name of the author, and document type. In addition, if the user wishes to use
the \texttt{acro} environment for managing the abbreviations and acronyms, these
must also be declared at the document preamble. Table~\ref{table:vars} lists
the variables and their use.



\begin{table}[h]\centering
\ra{1.3}
    \begin{tabularx}{\textwidth-1em}{p{6em}l>{\raggedleft\arraybackslash}X} \toprule
        Variable & Purpose & Mandatory  \\ \midrule
        \texttt{title} & The main title & Yes \\
        \texttt{author} & The author(s) & Yes \\
        \texttt{date} & Date of e.g. acceptance & Yes \\
        \texttt{level} & Document type, e.g. thesis or report & Yes \\
        \texttt{abstract} & The abstract & When description pages enabled \\
        \texttt{supervisors} & Supervisors of e.g. for thesis & No \\
        \texttt{assigned} & If assigned by a third party & No \\

        \bottomrule
    \end{tabularx}
\caption{List of variables that describe different aspects of the document.}
\label{table:vars}
\end{table}


\subsection{Sectioning and Structure}

The template uses standard convention for sectioning, i.e. 

\begin{verbatim}
\section{First Level}
\subsection{Second Level}
\subsubsection{Third Level}
\end{verbatim}

\noindent
As expected, the table of contents is generated automatically. This template creates
the list of figures, tables, and abbreviations by default. If you wish not to include
these sections, you can comment them out in the latex source.

\section{Tables and Images}


This section covers the basics of using both tables and images. It is likely
that you may need to adjust or alter the default parameters to suit your
specific need. In general, you should use search
engines to find what solutions others have used. It is likely that
what you are attempting to do has already been done.

\subsection{Tables}

Tables are an art form. It is generally advisable to omit as much lines
as possible while maintaining clarity. Table~\ref{table:web} is an example,
where spacing is used to separate the items. 

\begin{table}[h]\centering
\ra{1.3}
    \begin{tabularx}{\textwidth-1em}{llc} \toprule
        Abbreviation & Technology & Organization  \\ \midrule
        HTML 5.1 & Hypertext Markup Language & W3C \\
        CSS  & Cascading Style Sheets & W3C  \\
        HTTP/1.1& Hypertext Transfer Protocol &  IETF \\
        TLS 1.1 & Transport Layer Security & IETF \\
        JSON & JavaScript Object Notation & IETF \\
        ES & ECMAScript & Ecma International \\
        \bottomrule
    \end{tabularx}
\caption{Web standards and governing standardization organizations.}
\label{table:web}
\end{table}

\noindent
In many cases the example table is not adequate for your data. The table
may prove to be too narrow for large datasets. In cases like these you
should consider moving the table to a separate page, and changing the
orientation to landscape. For this you can try e.g. \texttt{sidewaystable} 
from the \texttt{rotating} package. There is also a package, \texttt{pdflscape}, that automatically
orients a single PDF page to landscape mode in PDF readers.


\subsection{Images}

Latex supports almost every common image and vector format with appropriate 
packages. Vector formats such as \texttt{.eps} or \texttt{.pdf} (with an embedded vector image)
are highly recommended.

For pictures that can not be vectorized, \texttt{.png} format is recommended. You
may also use \texttt{.jpeg} for photographs. Note, that if you require accurate
reproduction of colors in print, you may need to use CMYK colors, and worry about different gamuts.

\section{Mathematics}

Latex supports mathematic equations. Here are a few examples.

\begin{eqnarray}
    u_\alpha & = & \epsilon^2 \kappa_{xxx} 
    \left( y-\frac{1}{2}y^2 \right),
    \label{equ}  \\
    v & = & \epsilon^3 \kappa_{xxx} y\,,
    \label{eqv}  \\
    p & = & \epsilon \kappa_{xx}\,.
    \label{eqp}
\end{eqnarray}

\begin{eqnarray}
    \omega_1 & = &
    \frac{\partial w}{\partial y}-\frac{\partial v}{\partial z}\,,
    \nonumber  \\
    \omega_2 & = & 
    \frac{\partial u}{\partial z}-\frac{\partial w}{\partial x}\,,
    \label{eqcurl}  \\
    \omega_3 & = & 
    \frac{\partial v}{\partial x}-\frac{\partial u}{\partial y}\,.
    \nonumber
\end{eqnarray}

\section{Citations, Abbreviations, and Bibliography}

Here's a test picture.

\jamkfigure[2in]{src/img/test.png}{Test caption}{fig:dice}

Let's reference the test picture! Picture \ref{fig:dice} is a picture of
dice! 

If you didn't know, there are a lot of different types of dices. Table \ref{table:dicetypes} contains information about different kinds of dice!

\begin{table}[h]\centering
\ra{1.3}
    \begin{tabularx}{\textwidth-1em}{llc} \toprule
        \textbf{Type} & \textbf{Number of sides} & \textbf{Usage} \\ \midrule
        D4 & 4 & Tabletop RPGs \\
        D6 & 6 & Gambling, games... \\
        D10 & 10 & Tabletop RPGs \\
        D20 & 20 & Tabletop RPGs \\
        D100 & 100 & Tabletop RPGs \\
        \bottomrule
    \end{tabularx}
\caption{information
about different kinds of dice}
\label{table:dicetypes}
\end{table}

Let's cite a reference without any additional information: ~\cite{einstein}

Let's cite a reference some additional information (e.g. page number): ~\parencite[32]{einstein}

Someone might want to reference a acronym e.g. \ac{tcp} (see .tex file for details)

\subsection{lorem ipsum}

Suspendisse consequat lectus urna, vel hendrerit tortor euismod quis. Aliquam
diam urna, rhoncus at suscipit nec, pellentesque sed tellus. Fusce auctor
dignissim purus vel maximus. Morbi faucibus, lectus eget tempor posuere, tellus
ex imperdiet nibh, sed ultricies risus diam vitae nisi. Sed sed augue
malesuada, fringilla sem rutrum, molestie erat.  Phasellus non fringilla ex.
Integer eu lacinia est. Mauris commodo arcu eu consectetur condimentum. Donec
at ipsum at nisl blandit commodo a vitae nulla.  Quisque luctus sit amet turpis
vitae auctor. Nunc blandit metus ligula, nec auctor tortor sollicitudin et.
Nullam tristique efficitur ipsum, vel laoreet sapien lobortis at. Ut aliquet,
nulla id lobortis scelerisque, neque ante hendrerit turpis, vitae maximus sem
neque quis felis. Ut metus ante, sodales eu tincidunt eu, lobortis id mauris.

\subsubsection{more lorem ipsum}

Suspendisse consequat facilisis lacus, eget varius neque. Mauris blandit id
tellus vel consectetur. Integer porta tempor arcu, quis sodales urna posuere a.
Phasellus a lacinia dolor. Nam nec dui massa. Praesent vestibulum purus ac
felis volutpat vehicula. Sed sem nisl, hendrerit id gravida at, condimentum
hendrerit massa.  Maecenas vitae erat laoreet, semper enim sit amet,
condimentum ipsum.~\parencite{Puuska_2019,Haar_1910,Sipola_2020}

Donec at porttitor nibh. Suspendisse feugiat consequat ornare.  Mauris varius
porttitor libero ut facilisis. Pellentesque quis eros eros. Donec quis cursus
lorem. In eget diam felis. Sed dictum, tellus bibendum dictum commodo, ligula
felis semper nisl, ac molestie magna lacus vitae turpis. In volutpat nunc at
finibus vehicula. Vestibulum pretium at nibh in tempor. Cras sed mi sit amet
orci scelerisque mollis. Donec aliquet laoreet augue, ut malesuada massa semper
a. Suspendisse ac mi luctus, fringilla odio pellentesque, congue lorem.
Curabitur varius nunc eu elit mattis, sed gravida urna hendrerit. Duis eget
enim eget massa faucibus finibus. Suspendisse potenti. Interdum et malesuada
fames ac ante ipsum primis in faucibus.



%%% BACKMATTER

\clearpage

\printbibliography 

\clearpage

\appendix

\section*{Appendices}
\addcontentsline{toc}{section}{Appendices}

\subsection*{Appendix 1. Some text}

Instead of the boring and nonsensical lorem ipsum text, these appendices 
contain a few select fables by the ancient author Phaedrus.

\paragraph{Vulpes et Corvus}
Quae se laudari gaudent verbis subdolis,
serae dant poenas turpi paenitentia.
Cum de fenestra corvus raptum caseum
comesse vellet, celsa residens arbore,
vulpes invidit, deinde sic coepit loqui:
`O qui tuarum, corve, pinnarum est nitor!
Quantum decoris corpore et vultu geris!
Si vocem haberes, nulla prior ales foret'.
At ille, dum etiam vocem vult ostendere,
lato ore emisit caseum; quem celeriter
dolosa vulpes avidis rapuit dentibus.
Tum demum ingemuit corvi deceptus stupor. 
(Phaedr. 1.8)

\paragraph{Serpens Misericordi Nociua}
Qui fert malis auxilium, post tempus dolet.
Gelu rigentem quidam colubram sustulit
sinuque fouit, contra se ipse misericors;
namque, ut refecta est, necuit hominem protinus.
Hanc alia cum rogaret causam facinoris,
respondit: ``Ne quis discat prodesse improbis.''
(Phaedr. 4.20)



\clearpage

\subsection*{Appendix 2. Some text}

\paragraph{Vulpis et Draco}
Vulpes cubile fodiens dum terram eruit
agitque pluris altius cuniculos,
peruenit ad draconis speluncam ultimam,
custodiebat qui thesauros abditos.
Hunc simul aspexit: ``Oro ut inprudentiae
des primum ueniam; deinde si pulchre uides
quam non conueniens aurum sit uitae meae,
respondeas clementer: quem fructum capis
hoc ex labore, quodue tantum est praemium
ut careas somno et aeuum in tenebris exigas?''
``Nullum'' inquit ille, ``uerum hoc ab summo mihi
Ioue adtributum est.'' ``Ergo nec sumis tibi
nec ulli donas quidquam?'' ``Sic Fatis placet.''
``Nolo irascaris, libere si dixero:
dis est iratis natus qui est similis tibi.''
Abiturus illuc quo priores abierunt,
quid mente caeca miserum torques spiritum?
Tibi dico, auare, gaudium heredis tui,
qui ture superos, ipsum te fraudas cibo,
qui tristis audis musicum citharae sonum,
quem tibiarum macerat iucunditas,
obsoniorum pretia cui gemitum exprimunt,
qui dum quadrantes aggeras patrimonio
caelum fatigas sordido periurio,
qui circumcidis omnem inpensam funeris,
Libitina ne quid de tuo faciat lucri.
(Phaedr. 4.21)

\clearpage

\end{document}

